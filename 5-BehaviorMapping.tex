\chapter{Behavior Mapping}

\section{Computing Behavioral Embeddings}
\citet{mcinnes_umap_2020} \citet{mcinnes_hdbscan_2017} \citet{campello_density-based_2013}

\subsection{Supervised Disparate Embeddings}
\NOTE{
	We compute Supervised-UMAP embeddings separately for each experiment.
	Computing supervised embeddings is only possible for annotated experiments.
	This is useful for exploring behavioral sub-categories.
	For instance, one annotation category, e.g. "grooming", can be consisted of two different clusters in the behavioral embedding space.
	We can further investigate how high-level behavioral annotations contain different but similar behaviors.
}

\subsection{Supervised Joint Embeddings}
\NOTE{
	We compute Supervised-UMAP embeddings of all annotated experiments together.
	There is no specific benefit or use case (except visualization purposes) for computing this.
	The resulting embedding usually is not homogeneous in terms of fly experiments.
	Different flies do not mix well in the behavioral space.
}

\subsection{Unsupervised Disparate Embeddings}
\NOTE{
	For each experiment, we compute a usual unsupervised embedding separately.
}

\subsection{Unsupervised Joint Embedding}
\NOTE{
	We compute a single behavioral embedding using all experiments.
	The problem with this approach is that embedding space is too crowded and thus we can not find meaningful and homogeneous clusters right away.
	This embedding might be useful for visualization purposes.
}

\subsection{Semi-supervised Pair Embeddings}
\NOTE{
	This is the novel and most useful embedding approach that we finally utilize to label unannotated experiments using annotated ones.
	We compute an embedding for each annotated and unannotated pair.
	For example, if there are $N_A$ annotated experiments and $N_U$ unannotated experiments we compute $N_A \cdot N_U$ embeddings in total.
	There are number of advantageous of this approach.
	Especially when, an behavioral repertoire of an annotated an unannotated are similar, resulting embedding is easy to interpret and use for clustering etc.
}
\section{Soft Clustering of Behavioral Embeddings}
\NOTE{
	We always use soft clustering feature of HDBSCAN, since it is very beneficial to have a composite assignment for a data-point.
	For example, 0.7 grooming, 0.3 proboscis pumping may indicate that those two behaviors are simultaneously exhibited or a combination of both is exhibited etc.
	We can always take $\argmax$ if a categorical label is needed.
}

\subsection{Disparate Clustering}

\NOTE{
	If embedding is a disparate embedding, then we directly cluster each of them separately.
	If a joint embedding or pair embedding will be clustered, then experiments need to be extracted from the embedding space first and then they need to be clustered separately.
}

\subsection{Joint Clustering}
\NOTE{
	This is only applicable to joint and pair embeddings. We cluster all experiments in the behavioral embedding together.
}

\subsection{Crosswise Clustering}
\NOTE{
	This is again only applicable to joint and pair embeddings.
	For joint embeddings, we exclude a sub-group of experiments.
	For pair embeddings, we exclude one of the pair experiments.
	Then rest of the embedding is clustered and clusters are formed.
	Finally for each excluded embedding, soft cluster memberships vectors are computed based on the formed clusters.
}

\subsection{Mapping Clusters to Behavioral Categories}
\NOTE{
	If a clustering contains annotated experiments, we map that clusters in that clustering to a behavioral composition as follows ({\it subject to change, there are couple of alternatives})
	\begin{equation}
		m_\alpha = \frac{\textnormal{number of frames with annotation }\alpha \textnormal{ in the cluster}}{\textnormal{total number of frames with annotation }\alpha}.
	\end{equation}
	So for each cluster, we end up with a vector $\mathbf{m}=(m_\alpha)$, represent it behavioral composition.
}

\subsection{Computing Behavioral Scores}
\NOTE{
	Behavioral score of unannotated experiment will be computed using clustering membership score and behavioral composition mapping of those clusters.
	For example, using semi-supervised pair embeddings and crosswise clustering; one can compute behavioral scores for a frame as follows;
	\begin{equation}
		y_\alpha = \sum_{c=0}^C m^c_\alpha
	\end{equation}
	where $C$ is the number of clusters, $\mathbf{m}^c$ is the behavioral composition of cluster $c$.
	As a result, for each frame we end up with a behavioral score vector $\mathbf{y}=(y_\alpha)$.
}

\section{Nearest Neighbor Analysis and Classification}
\subsection{}
Consider two experiments, an annotated one $\exptAnn$, and an unannotated one $\exptUnann$, and their semi-supervised pair embeddings, respectively $\Ann{\V{E}}=\qmatrix{\Ann{\V{e}}_1, \cdots \Ann{\V{e}_{\Ann{F}}}}$ and $\Unann{\V{E}}=\qmatrix{\Unann{\V{e}}_1, \cdots \Unann{\V{e}}_{\Unann{F}}}$, where $\Ann{F}$ and $\Unann{F}$ are the total numbers of data points, e.g., frames, estimated as dormant and active.
Given the true annotations $\Ann{\V{y}}$ of $\exptAnn$, and $K$ behavioral categories, the goal is to compute $\V{\hat{b}}_f=\qmatrix{\hat{b}_{f,1}, \cdots \hat{b}_{f,K}}$, representing the weights (in other words similarity score) of each behavioral categories, for $\exptUnann$, using $\Ann{\V{E}}, \Unann{\V{E}}$ and $\Ann{\V{y}}$.

The procedure start by constructing $k$-nearest neighbor graph using $\Ann{\V{E}}$ and $\Unann{\V{E}}$, and $\NN{k}{f}$ denotes the set of indices of $\Unann{\V{e}}_f$'s $k$ nearest neighbors. Then, the initial $k$-NN weight $b_{f,i}$ for each query point (i.e., frame) $f$ of $\exptUnann$, and behavioral category $i$, is computed by
\begin{equation}
	b_{f,i} = \begin{dcases}
		\sum_{f^\prime \in \NNi{k}{i}{f}}\frac{1}{\dist{\Unann{\V{e}}_f}{\Ann{\V{e}}_{f^\prime}}^p + \epsilon} & \textnormal{if } \card{\NNi{k}{i}{f}} \neq 0, \\
		0                                                                                                      & \textnormal{if otherwise},
	\end{dcases} \qbwhere p \in \set{0,1,2}.
\end{equation}
Here, $\NNi{k}{i}{f}=\Set{f^\prime}{\Ann{y}_{f^\prime}=i, \dband f^\prime \in \NN{k}{f}}$, is the set of indices of data points of $\exptAnn$ whose annotation is behavior category $i$, and is one of the $k$ nearest neighbors of $\Unann{\V{e}}_f$.
$\dist{\Unann{\V{e}}_f}{\Ann{\V{e}}_{f^\prime}}$ is the euclidean distance between $\Unann{\V{e}}_f$ and $\Ann{\V{e}}_{f^\prime}$, and $p$ parameterizes the relation between distance and weight $b_{f,i}$.
We add a small number $\epsilon$ ($10^{{-}6}$) to the denominator to avoid numerical errors.
The resulting vector $\V{b}_f = \qmatrix{b_{f,1}, \cdots, b_{f,K}}$ weights the similarity of frame $f$ to each behavioral category in the shared embeddings space based on the nearest neighbors.

Naturally, the number of occurrences or durations of the behavior bouts are different for each behavioral category, and therefore, $\Ann{\V{y}}$ is highly imbalanced.
As a result, number of nearest neighbors and $\V{b}_f$ are biased in favor of frequently occurring and long-bout behaviors.
For instance, pumping-like movement of proboscis occur more frequently in longer bouts than switch-like movement of haltere.
Especially when the $k$ is large, it becomes crucial to consider the imbalanced distribution of behavior occurrences, since the embedding space will be dominated by frequent behaviors.
Thus, incorporating this imbalance to the formulation may help to improve the recall of rarely occurring short-bout behaviors and precision of frequently occurring long-bout behaviors.
To achieve this, we normalize previously computed scores $b_{f,i}$ by a function of number of occurrences of the behavioral category $i$ as follows:

\begin{equation}
	b^\prime_{f,i} = \frac{b_{f,i}}{\rbr{1 + N^{\plus}_i}^p} \qbor \frac{b_{f,i}}{\log_k(1 + N^{\plus}_i)} \qbwhere p \in \set{0, \sfrac{1}{2}, 1}, k \in \set{2, 10},
\end{equation}
where $N^{\plus}_i = \card*{\Set{f}{y^{\plus}_f=i}}$ is the number of frames annotated as behavioral category $i$.
In the above equation, two different alternatives are given for this normalization step; a polynomial one and a logarithmic one, where $p$ and $k$ parameterize the relation between $N^{\plus}_i$ and $b^\prime_{f,i}$.
For instance, if one is mostly interested in achieving high recall for frequently occurring behaviors, low $p$ values or using the logarithmic alternative might be more appropriate.
It may be even desired to set $p=0$, and not considering the number of occurrences in some cases, see Section~\ref{chapter:employing-proposed-pipeline} for more details.

The resulting vector $\V{b^{\prime}}_f \in \reals^K$ is dependent on the annotated experiment $\exptAnn$, and the vectors computed based on different annotated experiments are not comparable with each other.
Hence, we map values of $b^\prime_{f,i}$ to $\sqbr{0,1}$ by using either $\operatorname {softmax}$ function or $\textnormal{L}_1$ normalization as follows:
\begin{equation}
	\hat{b}_{f,i} = \frac{\exp{b^\prime_{f,i}}}{\sum_{j=1}^{K} \exp{b^\prime_{f,j}}} \qbor \frac{b^\prime_{f,i}}{\sum_{j=1}^{K} b^\prime_{f,j}}.
\end{equation}
The resulting behavioral weight vector $\V{\hat{b}}_f \in \sqbr{0,1}^K$ can be considered as a probability distribution.
Here, the vector $\V{\hat{b}}_f$ represents the behavioral characteristics of frame $f$ of $\exptUnann$ based on the behavioral repertoire of $\exptAnn$.
The voting-like scheme, which is used to finalize classification, incorporates all annotated experiments $\exptAnn_r$ considering corresponding $\V{\hat{b}}_r$ behavioral weight vectors.

\subsection{}

\begin{equation}
	v_{f,i} = \rbr{\log_2(K) - \entropy{\V{\hat{b}}_f}} \hat{b}_{f,i} \qbor \rbr{ 1 - \max_{1 \leq j \leq K} \hat{b}_{f,j}} \hat{b}_{f,i} \qbor \hat{b}_{f,i}
\end{equation}

\begin{equation}
	\V{v}_f = \qmatrix{v_{f,1}, \cdots, v_{f,K}}
\end{equation}
