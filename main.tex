\documentclass[
  12pt,
  oneside,
  a4paper
  ]{report}
%%%%%%%%%%%%
% Packages %
%%%%%%%%%%%%
\usepackage{inputenc}
\usepackage[
  rmargin=2.5cm,
  lmargin=3.5cm,
  tmargin=2.5cm,
  bmargin=2.5cm
  ]{geometry}
% \usepackage{showframe}
\usepackage{thesis}

%%%%%%%%%%%%%%%
% Definitions %
%%%%%%%%%%%%%%%
\definecolor{lacivert}{rgb}{0.0, 0.00, 0.52}

%%%%%%%%%%%%%%%%%%%%
% Defined Commands %
%%%%%%%%%%%%%%%%%%%%
% \DeclarePairedDelimiter{\norm}{\lVert}{\rVert}
% \DeclarePairedDelimiter{\abs}{\lvert}{\rvert}
\DeclarePairedDelimiter{\card}{\lvert}{\rvert}

\DeclareMathOperator*{\argmax}{arg\,max}
\DeclareMathOperator*{\argmin}{arg\,min}
\DeclareMathOperator{\distance}{d}
\DeclareMathSymbol{\omicron}{\mathord}{letters}{"6F}
\newcommand{\rangedots}{\mathinner{\ldotp \ldotp}}
\newcommand{\concat}{\; \middle| \;}
\newcommand\orient[1]{\operatorname{orient}(#1)}
\newcommand\rbr[1]{\left(#1\right)}
\newcommand\sqbr[1]{\left[#1\right]}
\newcommand\cbr[1]{\left\{#1\right\}}
\newcommand\abr[1]{\langle#1\rangle}
\newcommand\set[1]{\left\{ \, #1 \, \right\}}
\newcommand\Set[2]{\left\{ \, #1 \; \colon \; #2 \, \right\}}
\newcommand{\V}[1]{\symrm{\symbf{#1}}}
\newcommand{\dist}[2]{\distance \left( #1, #2 \right)}
\newcommand{\NN}[3]{{#1\mathrm{NN}}_#2 \left( #3 \right)}
\newcommand{\function}[2]{\operatorname{#1} \left( #2 \right)}
\newcommand{\g}[1]{\operatorname{s} \left( #1 \right)}
\newcommand{\minus}{\scalebox{0.8}{-}}
\newcommand{\plus}{\scalebox{0.6}{+}}
\newcommand{\Ann}[1]{#1^{\plus}}
\newcommand{\Unann}[1]{#1^{\minus}}
\newcommand{\exptAnn}{\Ann{\symrm{expt}}}
\newcommand{\exptUnann}{\Unann{\symrm{expt}}}
\newcommand{\entropy}[1]{\symrm{H}\rbr{#1}}
\newcommand{\qmatrix}[1]{\begin{bmatrix} #1 \end{bmatrix}}
\newcommand{\qbor}{\quad \textnormal{or} \quad}
\newcommand{\qbwhere}{\quad \textnormal{where} \quad}
\newcommand{\dband}{\; \textnormal{and} \;}

\let\bsim\sim 
\renewcommand{\sim}{{\bsim}}
\renewcommand{\d}[1]{\ensuremath{\operatorname{d}\!{#1}}}

\newcommand{\TODO}[1]{\textcolor{red}{[\texttt{TODO:} #1]}}
\newcommand{\NOTE}[1]{\textcolor{lacivert}{[\texttt{NOTE:} #1]}}
\newcommand{\CITE}{\textcolor{purple}{[\texttt{CITE}]}}

%%%%%%%%%%%%
% Settings %
%%%%%%%%%%%%
\bibliographystyle{unsrtnat}

%%%%%%%%%%%%%%%%%%%%%%%%
% Thesis Main Document %
%%%%%%%%%%%%%%%%%%%%%%%%
\begin{document}

\title{}
\author{Ali Osman Berk Şapcı}
\maketitle

\tableofcontents
\listoffigures
\listoftables

\begin{abstract}
	A \textit{Bastı} (English: \textit{Basty}, Azerbaijani Turkish: \textit{Basdı}, Anatolian Turkish: \textit{Basdırık}) is an evil spirit in Turkic mythology which rides on people's chests while they sleep, bringing on  nightmares.
	Bastı sits astride a sleeper's chest and becomes heavier until the crushing weight awakens the terrified and breathless dreamer.
	The victim awakes, unable to move under the Bastıs weight.
	It may also include lucid dreams.
	% There are different types of Bastı in Anatolia: Al-Bastı, Kara-Bastı, Kul-Bastı, Sarı-Bastı.
\end{abstract}

\setlength{\parindent}{0pt}
\chapter{Introduction}\label{chapter:introduction}
Sleep is an essential behavioral program conserved across the animal kingdom, in diverse species ranging from jellyfish to humans, whose function remains unknown \citep{campbell_animal_1984, nath_jellyfish_2017}.
In mammals, sleep consists of multiple stages marked by physiological changes, including reductions in muscle tone and distinct electrophysiological activity patterns in the brain \citep{corner_sleep_1977, sauer_dynamics_2003} In invertebrates, sleep has largely been studied as a unitary process and identified by bouts of consolidated immobility.
Thus, careful characterization of underlying changes in behavior and physiology is needed for understanding the functional role of the sleep, and characterizing distinct sleep stages in powerful genetic model systems such as \textit{Drosophila Melanogaster}.

A flurry of recent technical advances enabled automated quantification of behaviors, and this has opened up a new field of ``computational ethology'' \citep{anderson_toward_2014, datta_computational_2019}.
Particularly, the recent progress in deep learning lead to emergence of  methods for tracking animal motion, which provided the opportunity of studying naturalistic behavior at a resolution not previously imaginable \citep{pereira_quantifying_2020}.
Developments of pose estimation and tracking tools such as DeepLabCut \citep{mathis_deeplabcut_2018} and SLEAP \citep{pereira_fast_2019, pereira_sleap_2022} collecting spatio-temporal data about the animals possible.
However, this data only consists coordinate values in two or three dimensional spatial domain, depending on the employed tracking tool.
Quantification of rich and complex behavioral repertoires of the animals, considering its temporal structure and ambiguity, is a fundamentally difficult problem without a clear ground truth, as discussed in \citet{pereira_quantifying_2020}.

\NOTE{
	\begin{itemize}
		\item A deep sleep stage in {Drosophila} with a functional role in waste clearance: \citep{van_alphen_deep_2021}
		\item Most sleep does not serve a vital function: {Evidence} from {Drosophila} melanogaster: \citep{geissmann_most_2019}
		\item Automated analysis of long-term grooming behavior in {Drosophila} using a k-nearest neighbors classifier: \citep{qiao_automated_2018}
		\item Mapping the stereotyped behaviour of freely moving fruit flies: \citep{berman_mapping_2014}
		\item Systematic exploration of unsupervised methods for mapping behavior: \citep{todd_systematic_2017}
	\end{itemize}
}
\NOTE{
	\begin{itemize}
		\item Predictability and hierarchy in \textit{{Drosophila}} behavior: \citep{berman_predictability_2016}
		\item The manifold structure of limb coordination in walking {Drosophila}: \citep{deangelis_manifold_2019}
		\item The 103,200-arm acceleration dataset in the {UK} {Biobank} revealed a landscape of human sleep phenotypes: \citep{katori_103200-arm_2022}
	\end{itemize}
}
\NOTE{
	\begin{itemize}
		\item A dictionary of behavioral motifs reveals clusters of genes affecting {Caenorhabditis} elegans locomotion: \citep{brown_dictionary_2013}
		\item Deconstructing {Hunting} {Behavior} {Reveals} a {Tightly} {Coupled} {Stimulus}-{Response} {Loop}: \citep{mearns_deconstructing_2020}
		\item Continuous {Whole}-{Body} {3D} {Kinematic} {Recordings} across the {Rodent} {Behavioral} {Repertoire} - {SI}: \citep{marshall_continuous_2021}
		\item Mapping {Sub}-{Second} {Structure} in {Mouse} {Behavior}: \citep{wiltschko_mapping_2015}
		\item The {Mouse} {Action} {Recognition} {System} ({MARS}) software pipeline for automated analysis of social behaviors in mice: \citep{segalin_mouse_2021}
		\item Neural control of affiliative touch in prosocial interaction: \citep{wu_neural_2021}
		\item Partitioning variability in animal behavioral videos using semi-supervised variational autoencoders: \citep{whiteway_partitioning_2021}
		\item Structure of the {Zebrafish} {Locomotor} {Repertoire} {Revealed} with {Unsupervised} {Behavioral} {Clustering}: \citep{marques_structure_2018}
	\end{itemize}
}

\NOTE{
	\begin{itemize}
		\item {DeepEthogram}, a machine learning pipeline for supervised behavior classification from raw pixels: \citep{bohnslav_deepethogram_2021}
		\item B-{SOiD}, an open-source unsupervised algorithm for identification and fast prediction of behaviors: \citep{hsu_b-soid_2021}
		\item Ethoscopes: {An} open platform for high-throughput ethomics: \citep{geissmann_ethoscopes_2017}
		\item {JAABA}: interactive machine learning for automatic annotation of animal behavior: \citep{kabra_jaaba_2013}
		\item Simple {Behavioral} {Analysis} ({SimBA}) – an open source toolkit for computer classification of complex social behaviors in experimental animals: \citep{nilsson_simple_2020}
	\end{itemize}
}

\chapter{Experiments and Data Collection}\label{chapter:expt-data-collection}
In this section, we describe how the sleep experiments are conducted and fly video recordings data were collected.

\section{Sleep Experiments and Video Recordings}

\textit{The experimental data were collected by Dr. Mehmet Fatih Keles at Johns Hopkins University.}

A custom imaging setup was used to perform high-resolution characterization of sleep-related behaviors in flies. This set up includes a custom 3D printed chamber (7.2X4.3X2.4 mm [WxHxL]) that is placed in front of an IR sensitive (Flir) 30 FPS camera with telecentric lens (Edmund Optics).

\begin{figure}[ht!]
	\centering
	\includegraphics[width=0.75\linewidth]{figures/ExperimentalSetup.pdf}
	\caption[An illustration of the experimental setup which is used to perform high-resolution imaging of experiments.]{An illustration of the experimental setup which is used to perform high-resolution imaging of experiments.}
\end{figure}

Flies recorded from ZT10-ZT2 (16 hours) total.
Each chamber has a food port (1.5 mm diameter) that allows access to liquid food (2.5\% yeast, 2.5\% sugar).
Recording setup is in a light tight box and humidity control (60\%) is achieved via a humidifier plugged into a humidity control switch. Experimental flies are loaded to individual chambers at ZT8-ZT9 via mouth pipette or a small vacuum pump.
Individual chambers are sealed with a 7×7 mm acrylic windows.
Windows are coated with SigmaCote to prevent flies from ventral or dorsal postural positions. 5-7 day old female and male flies are used in the experiments.

\section{Pose Estimation and Tracking}
A recently developed deep neural network based software, DeepLabCut was used to achieve markerless pose estimation.
Over 20 body parts are first labeled in 1654 images from 28 animals (16 female, 12 male) to train the model with a 95/5 train/test split.
We used a ResNet-50 based neural network with a batch size of 4 and 200k training iterations.
Rest of the settings were kept default.
The resulting network has a test error of 3.67 pixels.

\begin{figure}[ht!]
	\centering
	\includegraphics[width=0.75\linewidth]{figures/FlyTrackedBodyParts.png}
	\caption[An example frame of the fruit fly placed in 3D printed chamber.] {An example video recording frame of the fruit fly placed in 3D printed chamber. Colorful markers indicate the tracked body parts.}
\end{figure}

\section{Behavior Annotations}

3 human annotators labeled 5 different behaviors (feeding, grooming, moving, haltere switch, proboscis pumping) across 11 videos (14 hours and 16 hours each, respectively for 7 wild type sleep experiments and 4 sleep deprivation experiments).
A single experiment is annotated by all 3 annotators to check rigor and overlap among annotators. We only used the annotations of a fly on when  annotators agreed at least on the 90\% of the labels.

\begin{figure}[ht!]
	\centering
	\begin{subfigure}[ht!]{0.95\linewidth}
		\centering\includegraphics[width=\linewidth]{figures/FlyF11-01182022175505_annotation_ethogram.pdf}
		\caption{Wild type.}
	\end{subfigure}%

	\centering
	\begin{subfigure}[ht!]{0.95\linewidth}
		\centering\includegraphics[width=\linewidth]{figures/FlyF52SD-11242021161943_annotation_ethogram.pdf}
		\caption{Sleep-deprived.}
	\end{subfigure}%
	\caption[Two example ethograms of annotated behaviors observed during the sleep experiments.]{Two example ethograms of annotated behaviors observed during the sleep experiments. Dark period starts at ZT12, and ends at ZT0.}
\end{figure}

\chapter{Feature Extraction}
For a single experiment data, i.e., single fruit fly recorded between ZT10 and ZT2 (zeitgeber time 10 and 2), feature extraction consists of four consecutive steps, where the input in one stage is the output of the previous one.
The input of the first step is the raw output signal of the tracking and pose estimation model, in our case, the output of DeepLabCut, a toolbox for markerless pose estimation.
The feature extraction steps are as follows:

\begin{enumerate}
	\item Constructing pose values and preprocessing; dealing with occluded body parts, alignment of different orientations, filtering and imputation.
	\item Computing spatio-temporal features, such as distances between body parts, velocity, angular velocity from body part positions.
	\item Computing dynamic postural features by extending spatio-temporal features to multiple time-scales using wavelet transformation and sliding window statistics.
	\item Computing normalized high-dimensional behavioral representations.
\end{enumerate}

Matrices $\V{X} \in \mathbb{R}^{T \times N}$ and $\V{Y} \in \mathbb{R}^{T \times N}$ denotes a multivariate time series for $x$ and $y$ cartesian components of two-dimensional video recordings that are collected for $N$ tracked body parts of the fly on $T$ consecutive time stamps by a pose estimation model.
This multivariate time series matrices $\V{X}$ and $\V{Y}$ are the raw input data that goes into the first step of the feature extraction.
Note that the number of body parts, $N$, must be the same among all experiments conducted with different fruit flies, but the number of time stamps, $T$, might differ.
Each column of the $\V{X}=[\V{x}_1, \cdots, \V{x}_N]^\top$ and $\V{Y}=[\V{y}_1, \cdots, \V{y}_N]^\top$ can be written respectively as follows;

\begin{align*}
	\V{y}_i & = (y_{i,1}, y_{i,2}, \dots, y_{i,t-1}, y_{i,t}, y_{i,t+1}, \dots, y_{i,T}), \\
	\V{x}_i & = (x_{i,1}, x_{i,2}, \dots, x_{i,t-1}, x_{i,t}, x_{i,t+1}, \dots, x_{i,T}).
\end{align*}

Here $i$ denotes the index of the body part, e.g., leg tip or proboscis.

In addition to $\V{X}$ and $\V{Y}$, a pose estimation model may report prediction scores for each tracked body part at each time step, which is the case for DeepLabCut as well.
$L \in \mathbb{R}^{N \times T}$ denotes the time series of prediction scores, each column of the $L=[\V{l_1}, \cdots, \V{l_N}]^\top$ can be written as follows;

\begin{equation*}
	\V{l}_i = (l_{i,1}, l_{i,2}, \dots, l_{i,t-1}, l_{i,t}, l_{i,t+1}, \dots, l_{i,T}).
\end{equation*}

The prediction scores tend to be very low when the body part is not visible.
Thus, $L$ provides valuable information about the occluded body parts.
In the Section~\ref{section:dealing-with-occluded-body-parts}, how $L$ is incorporated into construction of pose values is described in detail.

\section{Preprocessing}
The goal of in this step is to preprocess the signal, which involves filtering and imputation of certain video frames.
But in addition to this, there are a couple of optional procedures that can be beneficial for our task of learning stereotypical behaviors.
These additional procedures deal with occluded body parts of the fly, alignment of the fly orientations and defining new points of interest.

\subsection{Dealing with Occluded Body Parts}\label{section:dealing-with-occluded-body-parts}
As mentioned in the Section~\ref{section:challanges}, the two-dimensionality of the video recordings introduces a number of important challenges, and one of them is the problem of occluded body parts.
There are many types of occlusions, and some of them can be informally described as follows.
One examples is short occlusions resulting from postural changes.
Imputation of the time series $\V{X}$ and $\V{Y}$ for such short occlusions is relatively easy since the number of consecutive missing data points are not many.
However, this is not the case for long occlusions, which usually occur when the fly is dormant for a long period of time.
Especially for the body parts which have left and right counterparts, it is usual that the orientation of the fly results in one of the counterparts being occluded in long dormant epochs.
We follow multiple approaches to handle different type occlusions; namely imputation and elimination of corresponding data-points.
Before describing those approaches, we define a criterion for being occluded.

\subsubsection{Oriented Pose Values for Body Parts with Left \& Right Counterparts}
If the fly is oriented perpendicular to the camera perspective, as in Figure~\ref{figure:perpendicular-orientation}, then one of the left and right body parts is often occluded.
In other orientations (e.g., Figure~\ref{figure:parallel-orientation} and Figure~\ref{figure:oblique-orientation}),  both of them or none of them might be occluded as well.
However, in the conducted experiments, flies usually choose to stay dormant perpendicular to the camera perspective in long dormant epochs, as mentioned in Chapter~\ref{chapter:expt-data-collection}.
In such cases, one can concede to use only one of the left and right counterparts to construct pose values.
Therefore, this optional step is included in the behavioral mapping pipeline to reduce pose values of body parts with left and right counterparts to a single value.

We use prediction scores to determine which body part should be used to compute oriented pose values.
Let $i$ and $j$ be a pair of body parts which are left and right counterparts of each other, e.g., left haltere and right haltere.
Then, one can use the $\V{l}_i$ and $\V{l}_j$ to predict if one of them is occluded at a particular time step $t$.
Let $\orient{\V{x}_i, \V{x}_j}$ ($\orient{\V{y}_i, \V{y}_j}$) be a new pose vector which will be computed based on $\V{x}_i$ and $\V{x}_j$ ($\V{y}_i$ and $\V{y}_j$), e.g., a vector of oriented haltere pose values, composed of left haltere and right haltere pose values.
The following conditional procedures are proposed to compute oriented pose values from left and right pose values by deciding the orientation of the fly for a counterpart body pair. The procedures below can be used separately, or successively.

\begin{itemize}
	\item If $l_{i,t} - l_{j,t} \geq  \epsilon$, then, without loss of generality, $\orient{\V{x}_i, \V{x}_j}_t=x_{i,t}$ and $\orient{\V{y}_i, \V{y}_j}_t=y_{i,t}$ for a threshold $\epsilon$, typically $\epsilon > 0.5$.

	\item If $\card{\Set{t^{\prime}}{l_{i,t^{\prime}} > l_{j,t^{\prime}}, \, t^{\prime} \in [t-w \rangedots t+w]}} > w$, then, without loss of generality, $\orient{\V{x}_i, \V{x}_j}_t=x_{i,t}$ and $\orient{\V{y}_i, \V{y}_j}_t=y_{i,t}$, for a window of size $2w+1$.

	\item If $l_{i,t} > l_{j,t}$ and if the nearest confident left orientation is closer than the nearest confident right orientation, i.e., $$\argmin_{t^{\prime}} \Set{\abs{t-t^{\prime}}}{l_{i,t^{\prime}} - l_{j,t^{\prime}} \geq \epsilon} > \argmin_{t^{\prime}} \Set{\abs{t - t^{\prime}}}{l_{j,t^{\prime}} - l_{i,t^{\prime}} \geq \epsilon},$$ then, without loss of generality, $\orient{\V{x}_i, \V{x}_j}_t=x_{i,t}$ and $\orient{\V{y}_i, \V{y}_j}_t=y_{i,t}$.
	\item If simply $l_{i,t} > l_{j,t}$, then without loss of generality, $\orient{\V{x}_i, \V{x}_j}_t=x_{i,t}$ and $\orient{\V{y}_i, \V{y}_j}_t=y_{i,t}$.
\end{itemize}

Except directly comparing the prediction confidence scores as in the last procedure, some of the time points might be left with undecided orientations.
If the number of such time points is acceptable, then directly comparing the prediction scores for those time points is convenient and handy.

After applying the above procedures for a left and right counterpart pair $i$ and $j$, we can define oriented multivariate time series as;
\begin{align*}
	\V{X}^{\omicron} & = \rbr{\sqbr{\rbr{\V{x}_k}_{k \notin \bigcup_{\set{i,j} \in \mathcal{O}} \set{i,j}}}^\top \concat \sqbr{\rbr{\orient{\V{x}_i, \V{x}_j}}_{\set{i,j} \in \mathcal{O}}}^\top}, \\
	\V{Y}^{\omicron} & = \rbr{\sqbr{\rbr{\V{y}_k}_{k \notin \bigcup_{\set{i,j} \in \mathcal{O}} \set{i,j}}}^\top \concat \sqbr{\rbr{\orient{\V{y}_i, \V{y}_j}}_{\set{i,j} \in \mathcal{O}}}^\top},
\end{align*}
where $\mathcal{O}$ is the set of index pairs of left and right counterparts and $\bigcup_{\set{i,j} \in \mathcal{O}} \set{i,j}$ is the union of all indexes of such body parts pairs.
Applying described procedures for each left and right counterparts results in computing $\V{X}^{\omicron}$ and $\V{Y}^{\omicron}$.
This oriented versions of original data matrices can be used instead of $\V{X}$ and $\V{Y}$ in the rest of the pipeline, if desired.

\subsubsection{Detecting Occlusions Using Prediction Scores and Preternatural Tracking Predictions}

\subsection{Aligning Different Orientations}

\subsection{Filtering and Imputation}

\section{Computation of Spatio-temporal Features}
After preprocessing of pose values, it is now feasible to go one step further towards learning stereotypical behaviors.
Although tracking of relevant body parts and processing corresponding pose values is an essential step for quantifying behavior, a set of coordinate values is not sufficient to represent and capture complex spatio-temporal dynamics of animal behavior.
There are thousands of unique postures, and behaviors are not even exhibited by some static set of postures.
Instead, they are defined by expressive and meaningful spatio-temporal features such as distances, velocities, angles and angular velocities.
Therefor, one need to compute such features from the coordinate values of body parts in two-dimensional space.

The second stage of the feature extraction is computation of spatio-temporal features from pose values. Two type of features are computed in this stage, as listed below.
\begin{enumerate}
	\item \textbf{Snapshot features:} Spatio-temporal feature values computed at a snapshot of time, listed as follows:
	      \begin{itemize}
		      \item distances,
		      \item angles,
		      \item cartesian pose values (i.e., per body part features).
	      \end{itemize}
	\item \textbf{Gradient features:} Spatio-temporal feature values computed based how snapshot features change over time, listed as follows:
	      \begin{itemize}
		      \item change of distances,
		      \item change of angles (i.e., angular velocities),
		      \item change of cartesian pose values (i.e., body part velocities).
	      \end{itemize}
\end{enumerate}

The gradients of snapshot features are computed using second-order accurate central differences in the interior points.
The resulting gradient features have the same shape, i.e., the number of features and the number of time-stamps, as the snapshot features.

\subsection{Distances Between Body Parts}
Given a body part pair $\rbr{i,j}$, the distance between them at a time step $t$ is calculated as a usual Euclidean distance, given below.
\begin{equation}
	d^{i,j}_t = \sqrt{(x_{i,t} - x_{j,t})^2 + (y_{i,t} - y_{j,t})^2}.
\end{equation}

The corresponding gradient feature, namely the change of distance between body part $i$ and $j$, is computed using the second order gradient approximation,
\begin{equation}
	\dot{d}^{i,j}_t = \begin{cases} \frac{\abs{d^{i,j}_{t+1} - d^{i,j}_{t}}}{\Delta t} & \textnormal{if } $t=0$ \textnormal{ or } $t=T$, \\ \frac{\abs{d^{i,j}_{t+1} - d^{i,j}_{t-1}}}{2\Delta t} & \textnormal{otherwise}, \end{cases}
\end{equation}
where $\Delta t$ is the sampling period, and it is equal to $\sfrac{1}{\textnormal{FPS}}$ seconds.

\subsection{Joint Angles Between Body Parts}
Given a triplet of body parts $\rbr{i,j,k}$, angle between $i$ and $k$ around $j$ is computed using the 2-argument $\operatorname{arctangent}$ function as given below.
\begin{equation}
	w^{i,j,k}_t = \mathop{\rm{atan2}} \rbr{\det \begin{bmatrix} x_{i,t} - x_{j,t}, x_{k,t} - x_{j,t} \\ y_{i,t} - y_{j,t}, y_{k,t} - y_{j,t} \end{bmatrix}, \begin{bmatrix} x_{i,t} - x_{j,t} \\ y_{i,t} - y_{j,t} \end{bmatrix} \cdot \begin{bmatrix} x_{k,t} - x_{j,t}  \\ y_{k,t} - y_{j,t} \end{bmatrix}} + \pi.
\end{equation}

Then, similar to the change of distance features, angular velocities are approximated by,
\begin{equation}
	\dot{w}^{i,j,k}_t = \begin{cases} \frac{\abs{w^{i,j,k}_{t+1} - w^{i,j,k}_{t}}}{\Delta t} & \textnormal{if } $t=0$ \textnormal{ or } $t=T$, \\ \frac{\abs{w^{i,j,k}_{t+1} - w^{i,j,k}_{t-1}}}{2\Delta t} & \textnormal{otherwise}. \end{cases}
\end{equation}

\subsection{Cartesian Pose Values of Body Parts}
Cartesian pose values of a body part $i$ is straightforwardly given by the $x$ and $y$ coordinate values as follows,
\begin{align}
	x^{i}_t & = x_{i,t}, \\
	y^{i}_t & = y_{i,t}.
\end{align}

Note that for a single body part, two feature values are generated.
Corresponding gradient features, namely the body part velocities along each cartesian component, are computed by
\begin{align}
	\dot{x}^{i}_t & = \begin{cases} \frac{x^{i}_{t+1} - x^{i}_{t}}{\Delta t} & \textnormal{if } $t=0$ \textnormal{ or } $t=T$, \\ \frac{x^{i}_{t+1} - x^{i}_{t-1}}{2\Delta t} & \textnormal{otherwise}, \end{cases} \\
	\dot{y}^{i}_t & = \begin{cases} \frac{y^{i}_{t+1} - y^{i}_{t}}{\Delta t} & \textnormal{if } $t=0$ \textnormal{ or } $t=T$, \\ \frac{y^{i}_{t+1} - y^{i}_{t-1}}{2\Delta t} & \textnormal{otherwise}. \end{cases}
\end{align}

In order to compute overall two-dimensional velocity of a body part, one can always use the distance between origin and corresponding body part.

\subsection{Constructing Spatio-temporal Feature Matrices}
Let $\mathcal{C}, \mathcal{D}$, and $\mathcal{A}$ denote the sets of body parts, body part pairs, and body part triplets; respectively defining cartesian pose values, distances, and angles.
Similarly, let $\mathcal{C}^{\prime}, \mathcal{D}^{\prime}$, and $\mathcal{A}^{\prime}$ denote sets which define sets of respective gradient features.
Then snapshot feature matrix $S$ constructed as follows;
\begin{equation}
	\V{S} = \rbr{\rbr{x^i_t}_{1 \leq t \leq T, \, i \in \mathcal{C}} \concat \rbr{y^i_t}_{1 \leq t \leq T, \, i \in \mathcal{C}} \concat \rbr{d^{i,j}_t}_{1 \leq t \leq T, \, \cbr{i,j} \in \mathcal{D}} \concat \rbr{w_t^{i,j,k}}_{1 \leq t \leq T, \, \cbr{i,j,k} \in \mathcal{A}}} .
\end{equation}

Similarly, for gradient features, the feature matrix is constructed by concatenating change of distances, angular velocities and body part velocities; given by
\begin{equation}
	\V{G} = \rbr{\rbr{\dot{x}^i_t}_{1 \leq t \leq T, \, i \in \mathcal{C}^{\prime}} \concat \rbr{\dot{y}^i_t}_{1 \leq t \leq T, \, i \in \mathcal{C}^{\prime}} \concat \rbr{d^{i,j}_t}_{1 \leq t \leq T, \, \cbr{i,j} \in \mathcal{D}^{\prime}} \concat \rbr{\dot{w}_t^{i,j,k}}_{1 \leq t \leq T, \, \cbr{i,j,k} \in \mathcal{A}^{\prime}} }.
\end{equation}

The resulting two feature matrices are $\V{S} \in \mathbb{R}^{T \times \rbr{2\card{\mathcal{C}}+\card{\mathcal{D}}+\card{\mathcal{A}}}}$ and $\V{G} \in \mathbb{R}^{T \times \rbr{2\card{\mathcal{C}^{\prime}}+\card{\mathcal{D}^{\prime}}+\card{\mathcal{A}^{\prime}}}}$.
Let $N_S$ denote the number of snapshot features, being equal to $2\card{\mathcal{C}}+\card{\mathcal{D}}+\card{\mathcal{A}}$ and let $N_G$ denote the number of gradient features, which is equal to $2\card{\mathcal{C}^{\prime}}+\card{\mathcal{D}^{\prime}}+\card{\mathcal{A}^{\prime}}$.

\section{Computation of Dynamic Postural Features}\label{section:dynamic-postural-features}
Instantaneous values of spatio-temporal features do not provide a sufficient description of complex postural dynamics of behaviors.
Understanding the output of a complex biological system, in our case behavior, can only be achieved by studying multiple time-scales together.
Previous studies attempt to search behavioral motifs, e.g., repeated sub-sequences of actions with finite length, within the behavioral time series \CITE.
However, as \citet{berman_mapping_2014} states, this paradigm usually requires problems of temporal alignment and relative phasing between different scales.
Alternatively, extending spatio-temporal features to capture postural dynamics at different time-scales eliminate requirements of temporal alignment and motif based analysis.
In order to extend the spatio-temporal features to dynamic postural features, we applied wavelet transformation (similar to \citet{berman_mapping_2014}) and computed moving statistics at different time-scales (similar to \citet{kabra_jaaba_2013}), respectively for snapshot feature set ($\V{S}$) and gradient feature set ($\V{G}$).

\subsection{Moving Statistics of Gradient Features}
Gradient features only reflect the instantaneous values of velocities with respect to the sampling rate.
In order to capture how these values change within a given interval, the moving statistics, e.g., mean and standard deviation, of gradient features are computed with a sliding window approach.
Let $\tau$ be the window size parameter, i.e., the timescale of interest, then the moving mean of the corresponding gradient feature $\V{g}_{i}$ is given by the function $\mu_\tau$;
\begin{equation}
	\mu_{\tau}(g_{i,t}) = \sfrac{1}{\min\{t+\tau,T\}-\max\{t-\tau,1\}+1} \sum_{t^{\prime}=\max\{t-\tau,1\}}^{\min\{t+\tau,T\}} g_{i, t^{\prime}}.
\end{equation}

Similarly, the moving standard deviation of a gradient feature $g_{i,t}$ by is computed by $\sigma_\tau$, as in the below equation.
\begin{equation}
	\sigma_\tau(g_{i,t}) = \rbr{\sfrac{1}{\min\{t+\tau,T\}-\max\{t-\tau,1\}+1} \sum_{t^{\prime}=\max\{t-\tau,1\}}^{\min\{t+\tau,T\}} \rbr{\mu_{\tau}(g_{i,t}) - g_{i, t^{\prime}}}^2}^{\sfrac{1}{2}}
\end{equation}

Moving statistics feature generation approach is used to learn animal behavior by \citet{kabra_jaaba_2013}, and \citet{marshall_continuous_2021} is also included such features into the analysis.

\subsection{Wavelet Transformation of Snapshot Features}
The wavelet domain is a useful representation of postural dynamics due to the following reasons given by \citet{berman_mapping_2014}, and proposed spectrogram generation is used by others as well \citep{marshall_continuous_2021, todd_systematic_2017}.
\begin{itemize}
	\item It describes dynamics over multiple time-scales simultaneously by possessing a multi-resolution time-frequency trade-off.
	\item It eliminates the requirement of precise temporal alignment for capturing periodic behaviors by taking amplitudes of the continuous wavelet transform of each snapshot feature at different scales.
\end{itemize}
Given a function $s(t)$, continuous wavelet transformation at a frequency $f>0$ is expressed by the following integral;
\begin{equation}
	W_{f,t^{\prime}}\sqbr{s\rbr{t}} = \frac{1}{\sqrt{a\rbr{f}}} \int_{-\infty}^{\infty} s\rbr{t} \mathit{\Psi}^{\ast}\rbr{\frac{t - t^{\prime}}{a(f)}} \dd{t},
\end{equation}
where $\Psi$ is the wavelet function and $a$ is a function for converting frequencies to wavelet scale factor.
The Morlet wavelet is suitable for describing postural dynamics which is closely related to human perception, both hearing and vision \citep{daugman_uncertainty_1985}, and is used in the pipeline.
The corresponding wavelet function is given by
\begin{equation}
	\Psi(t) = \exp{\frac{t^2}{2}} \cos(w_0t),
\end{equation}
where $w_0$ is a non-dimensional parameter. The frequency to scale conversion function $a$ for Morlet wavelet is as follows;
\begin{equation}
	a\rbr{f} = \frac{w_0 + \sqrt{2+w_0^2}}{4 \pi f},.
\end{equation}

For the discrete sequence of snapshot feature $\V{s}_i$ with sampling period $\Delta t$, $W_{f,t^\prime}[s(t)]$ translates into;
\begin{equation}
	\label{eq:wavelet-practical}
	W_f\rbr{\V{s}_i, t^\prime} = \frac{1}{\sqrt{a(f)}} \sum_{t=1}^{T} {\Delta t} s_{i,t} \mathit{\Psi}^{\ast}\rbr{\frac{t-t^{\prime}}{a{f}}},
\end{equation}
where $t^\prime, t \in \mathbb{Z}$ and $ 1 \leq t^\prime \leq T$ \citep{torrence_practical_1998}.

\subsubsection{Normalization of Wavelet Power Spectrum}
In order to ensure that wavelet transforms (Equation~\ref{eq:wavelet-practical}) at each frequency $f$ are directly comparable to each other and to the other transformed time series, the transformation $W_f$ has to be normalized at each frequency $f$ to have unit energy.
This normalization for Morlet wavelet at frequency $f$ is given by
\begin{equation}
	C(f) = \frac{\pi^{-\frac{1}{4}}}{\sqrt{2a(f)}}\exp{\sfrac{1}{4}\rbr{w_0-\sqrt{w_0^2+2}}^2}.
\end{equation}
So resulting normalized transformation which is also used to generate the spectrogram in \citet{berman_mapping_2014} is given by
\begin{equation}
	W^0_f\rbr{\V{s}_i, t^\prime} = \frac{1}{C\rbr{f}} \abs{W_f\rbr{\V{s}_i, t^\prime}}.
\end{equation}

In addition to the above conventionally used normalization, we alternatively adopted the normalization proposed by \citet{liu_rectification_2007}. According to this alternative adjustment, the wavelet power spectrum should be equal to the transform coefficient squared divided by the scale it associates.
\begin{equation}
	W^{0}_f\rbr{\V{s}_i, t^\prime} = \frac{W_f\rbr{\V{s}_i, t^\prime}^2}{a\rbr{f}}
\end{equation}
We observed substantial improvements using this power spectrum (see Section~\ref{section:employing-proposed-pipeline}).

\subsubsection{Determining Spectrum Frequencies}
We investigate two different approaches for computing a set of frequencies and, we include both of them in the behavior mapping pipeline.
One set is dyadically spaced frequencies between $f_{\textnormal{min}}$ and $f_{\textnormal{max}}$ via;
\begin{equation}
	f_i = f_{\textnormal{max}} 2^{-\frac{i-1}{N_f-1}\log{\frac{f_{\textnormal{max}}}{f_{\textnormal{min}}}}},
\end{equation}
where $f_{\textnormal{max}} = \sfrac{FPS}{2} \textnormal{ Hz}$ is the Nyquist frequency.

The other alternative set of frequencies is linearly spaced between $f_{\textnormal{min}}$ and $f_{\textnormal{max}}$ by
\begin{equation}
	f_i = f_{\textnormal{min}} + \frac{f_{\textnormal{max}} - f_{\textnormal{min}}}{N_f-1}i,
\end{equation}
for $i=1,2,\dots,N_f$, and their corresponding wavelet scales are computed via $a\rbr{f_i}$.

\subsection{Constructing Dynamic Postural Feature Tensors}
Let $\mathcal{T}_S=\set{\sfrac{1}{f_{\textnormal{min}}}, \cdots \sfrac{1}{f_{\textnormal{max}}}}$ and $\mathcal{T}_G=\set{\tau_{\textnormal{min}}, \cdots, \tau_{\textnormal{max}}}$ denote the time-scale sets respectively for wavelet transforms of snapshot features and moving statistics of gradient features. Then corresponding feature tensors are given as follows:
\begin{align}
	\V{W}        & = \sqbr{W^0_{\sfrac{1}{f}}\rbr{\V{s}_i, t}}_{1 \leq i \leq N_S, \,1 \leq t \leq T, \, \sfrac{1}{f} \in \mathcal{T}_S}, \\
	\V{M}_\mu    & = \sqbr{\mu_\tau\rbr{g_{i,t}}}_{1 \leq i \leq  N_G,  \, 1\leq t \leq T, \, \tau \in \mathcal{T}_G},                    \\
	\V{M}_\sigma & = \sqbr{\sigma_\tau\rbr{g_{i,t}}}_{1 \leq i \leq N_G, \, 1\leq t \leq T, \, \tau \in \mathcal{T}_G},
\end{align}
where $\V{S} \in \mathbb{R}^{T \times N_S}$ and $\V{G} \in \mathbb{R}^{T \times N_G}$.

The resulting extended feature tensors of dynamic postural representations are $\V{W} \in \mathbb{R}^{T \times \card{\mathcal{T}_s} \times N_S}$, $\V{M}_\mu \in \mathbb{R}^{T \times \card{\mathcal{T}_v} \times N_G}$ and $\V{M}_\sigma \in \mathbb{R}^{T \times \card{\mathcal{T}_v} \times N_G}$.

\section{Computation of Behavioral Representations}
After applying wavelet transformation or computing moving statistics to extend extracted spatio-temporal features to dynamic postural features, a couple of additional operations are required to continue in the behavior mapping pipeline.

\subsection{Flattening Dynamic Postural Feature Tensors}
As constructed in Section~\ref{section:dynamic-postural-features}, dynamic postural feature tensors are $\V{W} \in \mathbb{R}^{T \times \card{\mathcal{T}_S} \times N_S}$, $\V{M}_\mu \in \mathbb{R}^{T \times \card{\mathcal{T}_G} \times N_G}$, and  $\V{M}_\sigma \in \mathbb{R}^{T \times \card{\mathcal{T}_G} \times N_G}$.
In order to apply manifold learning-based dimensionality reduction algorithms or traditional machine learning algorithms such as decision trees, the last two dimensions of these feature tensors needs to be flattened.
As a result, feature matrices are $\V{W}^\ast \in \mathbb{R}^{T \times \rbr{ N_S\card{\mathcal{T}_S}}}, \V{M}_\mu^\ast \in \mathbb{R}^{T \times \rbr{ N_G\card{\mathcal{T}_G}}}$ and $\V{M}_\sigma^\ast \in \mathbb{R}^{T \times \rbr{ N_G\card{\mathcal{T}_G}}}$ are obtained.

\subsection{\texorpdfstring{$\textnormal{L}_1$}{L1} Normalization of Frames}
Dynamic postural feature distributions of similar behaviors may differ among flies due to different characteristics such as gender, and being sleep-deprived.
Due to the two-dimensional nature of the video recordings, different orientations may cause observing different feature values for the same behavior.
In order to have a homogeneous feature space among flies and along the time, at each time step $t$, $\textnormal{L}_1$ normalization is applied as follows;
\begin{align}
	\V{\hat{w}}_i & = \rbr{\frac{w^\ast_{t,i}}{\sum_{j=1}^{N_S \card{\mathcal{T}_s}}w^\ast_{t, j}}}_{1 \leq t \leq T}, \\
	\V{\hat{W}}   & = \sqbr{\rbr{\V{w}_i}_{1 \leq i \leq N_S\card{\mathcal{T}_S}}}.
\end{align}
Similarly, $\textnormal{L}_1$ normalized versions of $\V{M}^\ast_\mu$ and $\V{M}^\ast_\sigma$, namely $\V{\hat{M}}_\mu$ and $\V{\hat{M}}_\sigma$ are obtained.
Here $\V{\hat{W}}$, $\V{\hat{M}}_\mu$ and $\V{\hat{M}}_\sigma$ are the final multivariate time series of normalized high dimensional behavioral representation of a single experiment data, i.e., single fruit fly recorded between ZT10 and ZT2. Notice that, we may treat each time step, i.e., frame, as a discrete probability distribution after $\textnormal{L}_1$ normalization.

\section{Summary}

\chapter{Experiment Outlining}

\section{Overview of Experiment Outlining}
Each experiment comprises sixteen hours of video recording spanning both awake and asleep epochs.
Since we are only interested in the behavioral repertoire exhibited during sleep, time intervals where the animal is dormant, namely the dormancy epochs, should be detected before proceeding with the behavior mapping stage.
We characterize dormancy epochs by lack of macro-activities, i.e., significant postural and locational changes, which can be qualified by displacement of the animal.
After detecting and excluding intervals of macro-activities, we end up with time points where the fly is dormant.
Yet, we need further process the dormancy epochs, as we are not interested in the time points where the fly is totally quiescent.
Our major focus is on the micro-activity bouts manifested during a dormancy epoch.
In order to detect those bouts, we should distinguish micro-activities exhibited during dormant epochs from macro-activities by quantifying them with a closer look at various body parts. We use term ``bouts'' and ``epochs'' respectively for micro-activities and macro-activities to reflect their difference in terms of duration. As discussed in Section~\ref{section:analyzing-behavioral-repertoire}, behaviors categorized in micro-activities are tends to have shorter durations, compared to macro-activities.

In this stage, our ultimate goal is to extract bouts of micro-activities exhibited during the dormancy state.
Extracted micro-activity bouts constitute the data points that are subject to behavior mapping.
There are several benefits of reducing the data points to this subset of dormancy and micro-activity, compared to using the entire experiment for behavior mapping.
Considering the high frame rate and long length of video recordings, computational requirements is an important concern in our pipeline.
Since roughly at least $90\%$ of the frames are either totally quiescent or macro-activities, e.g., walking, this approach has the benefit of reducing the computational requirements significantly.
Another critical point is that, quiescent, e.g., rest, frames contain only noise energy, and normalizing each frame amplifies me normalization amplifies this low-level noise energy, generating a uniform-like probability distribution for behavioral representation \citep{todd_systematic_2017}.
Eliminating totally quiescent frames without any micro-activity avoids this.
Also, as we are only interested in the behaviors exhibited during sleep, excluding macro-activity frames prevent domination of large number of frames with walking and macro-activities in the behavioral embedding space.

\NOTE{Overview of sections.}

\section{Quantifying Activities}
\subsection{Dormancy and Macro-activity Epochs}
When the fly is awake, many behaviors are manifested by featuring major postural change and displacement in different ways.
We categorize this type of behaviors under the umbrella term of macro-activity, and dormancy is defined as the lack of macro-activities, covering micro-activities.
One can characterize macro-activities without considering its sub-categories by using the velocities, i.e., gradient features.
In order to distinguish sub-categories of macro-activities, such as walking and rearing, more detailed and descriptive features are required.
However, in our case, computing a single scalar value to capture overall movement of the fly results in sufficient performance for detecting macro-activity epochs. We define this feature value by summing gradient features for all time scales as follows:
\begin{equation}
	v_t = \sum_{\tau \in \mathcal{T}_G}{\sum_{i=1}^{N_G}{\V{M}^\mu_{t,\tau,i}}},
\end{equation}
where resulting velocity-based feature vector is $\V{v}=\qmatrix{v_1, \cdots, v_T}$. Essentially, high and low values of $v_t$ indicate macro-activity and dormancy, respectively. Micro-activities can not be detected using such a straightforward and general value, and therefore, can not be distinguished from dormancy by using only this quantity.

\subsection{Micro-activity Bouts}

\begin{equation}
	u_{t,i} = \sum_{\sfrac{1}{f} \in \mathcal{T}_G}{\V{W}_{t,f,i}}
\end{equation}

\section{Detecting Activities}

\subsection{Unsupervised Approach}

\subsection{Weakly Supervised Approach}

\chapter{Behavior Mapping}

\section{Computing Behavioral Embeddings}
\citet{mcinnes_umap_2020} \citet{mcinnes_hdbscan_2017} \citet{campello_density-based_2013}

\subsection{Supervised Disparate Embeddings}
\NOTE{
	We compute Supervised-UMAP embeddings separately for each experiment.
	Computing supervised embeddings is only possible for annotated experiments.
	This is useful for exploring behavioral sub-categories.
	For instance, one annotation category, e.g. "grooming", can be consisted of two different clusters in the behavioral embedding space.
	We can further investigate how high-level behavioral annotations contain different but similar behaviors.
}

\subsection{Supervised Joint Embeddings}
\NOTE{
	We compute Supervised-UMAP embeddings of all annotated experiments together.
	There is no specific benefit or use case (except visualization purposes) for computing this.
	The resulting embedding is usually not homogeneous in terms of fly experiments.
	Different flies do not mix well in the behavioral space.
}

\subsection{Unsupervised Disparate Embeddings}
\NOTE{
	For each experiment, we compute a usual unsupervised embedding separately.
}

\subsection{Unsupervised Joint Embedding}
\NOTE{
	We compute a single behavioral embedding using all experiments.
	The problem with this approach is that embedding space is too crowded, and thus we can not find meaningful and homogeneous clusters right away.
	This embedding might be useful for visualization purposes.
}

\subsection{Semi-supervised Pair Embeddings}
\NOTE{
	This is the novel and most useful embedding approach that we finally utilize to label unannotated experiments using annotated ones.
	We compute an embedding for each annotated and unannotated pair.
	For example, if there are $N_A$ annotated experiments and $N_U$ unannotated experiments, we compute $N_A \cdot N_U$ embeddings in total.
	There are number of advantageous of this approach.
	Especially when the behavioral repertoire of the annotated and the unannotated are similar, resulting embedding is easy to interpret and use for clustering, etc.
}
\section{Soft Clustering of Behavioral Embeddings}
\NOTE{
	We always use soft clustering feature of HDBSCAN, since it is very beneficial to have a composite assignment for a data-point.
	For example, 0.7 grooming, 0.3 proboscis pumping may indicate that those two behaviors are simultaneously exhibited or a combination of both is exhibited etc.
	We can always take $\argmax$ if a categorical label is needed.
}

\subsection{Disparate Clustering}

\NOTE{
	If embedding is a disparate embedding, then we directly cluster each of them separately.
	If a joint embedding or pair embedding will be clustered, then experiments need to be extracted from the embedding space first and then they need to be clustered separately.
}

\subsection{Joint Clustering}
\NOTE{
	This is only applicable to joint and pair embeddings. We cluster all experiments in the behavioral embedding together.
}

\subsection{Crosswise Clustering}
\NOTE{
	This is again only applicable to joint and pair embeddings.
	For joint embeddings, we exclude a subgroup of experiments.
	For pair embeddings, we exclude one of the pair experiments.
	Then rest of the embedding is clustered and clusters are formed.
	Finally, for each excluded embedding, soft cluster membership vectors are computed based on the formed clusters.
}

\subsection{Mapping Clusters to Behavioral Categories}
\NOTE{
	If a clustering contains annotated experiments, we map that clusters in that clustering to a behavioral composition as follows ({\it subject to change, there are couple of alternatives})
	\begin{equation}
		m_\alpha = \frac{\textnormal{number of frames with annotation }\alpha \textnormal{ in the cluster}}{\textnormal{total number of frames with annotation }\alpha}.
	\end{equation}
	So for each cluster, we end up with a vector $\mathbf{m}=(m_\alpha)$, represent it behavioral composition.
}

\subsection{Computing Behavioral Scores}
\NOTE{
	Behavioral score of unannotated experiment will be computed using clustering membership score and behavioral composition mapping of those clusters.
	For example, using semi-supervised pair embeddings and crosswise clustering; one can compute behavioral scores for a frame as follows;
	\begin{equation}
		y_\alpha = \sum_{c=0}^C m^c_\alpha
	\end{equation}
	where $C$ is the number of clusters, $\mathbf{m}^c$ is the behavioral composition of the cluster $c$.
	As a result, for each frame, we end up with a behavioral score vector $\mathbf{y}=(y_\alpha)$.
}

\section{Nearest Neighbor Analysis and Classification}
\subsection{Behavioral Weights}
Consider two experiments, an annotated one $\exptAnn$, and an unannotated one $\exptUnann$, and their semi-supervised pair embeddings, respectively $\Ann{\V{E}}=\qmatrix{\Ann{\V{e}}_1, \cdots \Ann{\V{e}_{\Ann{F}}}}$ and $\Unann{\V{E}}=\qmatrix{\Unann{\V{e}}_1, \cdots \Unann{\V{e}}_{\Unann{F}}}$, where $\Ann{F}$ and $\Unann{F}$ are the total numbers of data points, for example, frames, estimated as dormant and active.
Given the true annotations $\Ann{\V{y}}$ of $\exptAnn$, and $K$ behavioral categories, the goal is to compute $\V{\hat{b}}_f=\qmatrix{\hat{b}_{f,1}, \cdots \hat{b}_{f,K}}$, representing the weights (in other words, the similarity score) of each behavioral category for $\exptUnann$, using $\Ann{\V{E}}, \Unann{\V{E}}$ and $\Ann{\V{y}}$.

The procedure start by constructing $k$-nearest neighbor graph using $\Ann{\V{E}}$ and $\Unann{\V{E}}$, and $\NN{k}{f}$ denotes the set of indices of $\Unann{\V{e}}_f$'s $k$ nearest neighbors. Then, the initial $k$-NN weight $b_{f,i}$ for each query point (i.e., frame) $f$ of $\exptUnann$, and behavioral category $i$, is computed by
\begin{equation}
	b_{f,i} = \begin{dcases}
		\sum_{f^\prime \in \NNi{k}{i}{f}}\frac{1}{\dist{\Unann{\V{e}}_f}{\Ann{\V{e}}_{f^\prime}}^p + \epsilon} & \textnormal{if } \card{\NNi{k}{i}{f}} \neq 0, \\
		0                                                                                                      & \textnormal{if otherwise},
	\end{dcases} \qbwhere p \in \set{0,1,2}.
\end{equation}
Here, $\NNi{k}{i}{f}=\Set{f^\prime}{\Ann{y}_{f^\prime}=i, \dband f^\prime \in \NN{k}{f}}$, is the set of indices of data points of $\exptAnn$ whose annotation is the behavior category $i$ and is one of the $k$ nearest neighbors of $\Unann{\V{e}}_f$.
$\dist{\Unann{\V{e}}_f}{\Ann{\V{e}}_{f^\prime}}$ is the euclidean distance between $\Unann{\V{e}}_f$ and $\Ann{\V{e}}_{f^\prime}$, and $p$ parameterizes the relation between distance and weight $b_{f,i}$.
We add a small number $\epsilon$ ($10^{{-}6}$) to the denominator to avoid numerical errors.
The resulting vector $\V{b}_f = \qmatrix{b_{f,1}, \cdots, b_{f,K}}$ weights the similarity of the frame $f$ to each behavioral category in the shared embedding space based on nearest neighbors.

Naturally, the number of occurrences or durations of the behavior bouts are different for each behavioral category, and therefore, $\Ann{\V{y}}$ is highly imbalanced.
As a result, number of nearest neighbors and $\V{b}_f$ are biased in favor of frequently occurring and long-bout behaviors.
For instance, pumping-like movements of the proboscis occur more frequently in longer bouts than switch-like movements of the haltere.
Especially when $k$ is large, it becomes crucial to consider the imbalanced distribution of behavior occurrences, since the embedding space will be dominated by frequent behaviors.
Thus, incorporating this imbalance into the formulation may help to improve the recall of rarely occurring short-bout behaviors and precision of frequently occurring long-bout behaviors.
To achieve this, we normalize the scores previously computed, $b_{f,i}$, as a function of the number of occurrences of the behavioral category $i$ as follows:

\begin{equation}
	b^\prime_{f,i} = \frac{b_{f,i}}{\rbr{1 + N^{\plus}_i}^p} \qbor \frac{b_{f,i}}{\log_k(1 + N^{\plus}_i)} \qbwhere p \in \set{0, \sfrac{1}{2}, 1}, k \in \set{2, 10},
\end{equation}
where $N^{\plus}_i = \card*{\Set{f}{y^{\plus}_f=i}}$ is the number of frames annotated as behavioral category $i$.
In the above equation, two different alternatives are given for this normalization step; a polynomial one and a logarithmic one, where $p$ and $k$ parameterize the relation between $N^{\plus}_i$ and $b^\prime_{f,i}$.
For instance, if one is mostly interested in achieving high recall for frequently occurring behaviors, low $p$ values or using the logarithmic alternative might be more appropriate.
It may be even desired to set $p=0$, and not considering the number of occurrences in some cases, see Section~\ref{chapter:employing-proposed-pipeline} for more details.

The resulting vector $\V{b^{\prime}}_f \in \reals^K$ is dependent on the annotated experiment $\exptAnn$, and the vectors computed based on different annotated experiments are not comparable with each other.
Hence, we map the values of $b^\prime_{f,i}$ to $\sqbr{0,1}$ using either the $\operatorname {softmax}$ function or $\textnormal{L}_1$ normalization as follows:
\begin{equation}
	\hat{b}_{f,i} = \frac{\exp{b^\prime_{f,i}}}{\sum_{j=1}^{K} \exp{b^\prime_{f,j}}} \qbor \frac{b^\prime_{f,i}}{\sum_{j=1}^{K} b^\prime_{f,j}}.
\end{equation}
The resulting behavioral weight vector $\V{\hat{b}}_f \in \sqbr{0,1}^K$ can be considered as a probability distribution.
Here, the vector $\V{\hat{b}}_f$ represents the behavioral characteristics of the frame $f$ of $\exptUnann$ based on the behavioral repertoire of $\exptAnn$.
The voting-like scheme, as described in Section~\ref{section:committee-voting}, incorporates the behavioral weight vectors of all annotated experiments to finalize the classification for $\exptUnann$.

\subsection{Experiment Committee by Voting}\label{section:committee-voting}
Consider all experiments: unannotated experiments $\exptUnann_1, \cdots, \exptUnann_{\Unann{R}}$, and annotated experiments $\exptAnn_1, \cdots, \exptAnn_{\Ann{R}}$, where $\Unann{R}$ and $\Ann{R}$ are the number of experiments, respectively.
The goal is to combine the behavioral weights of an unannotated experiment $\exptUnann_k$, calculated separately for each annotated experiment.

Let $\V{\hat{b}}_f^{k,l}$ denote the behavioral weights of $\exptUnann_k$ computed with $\exptAnn_l$. Each annotated experiment contributes to the overall behavioral score of $\exptUnann_k$; in other words, a committee consisting of annotated experiments vote according to $\V{\hat{b}}_f^{k,l}$. Before describing hard voting and soft voting approaches, there is one more step to discuss.

There exists a significant variation among the exhibited behavioral repertoires in the experiments.
An annotated experiment might lack some behavioral expressions or manifest some behaviors excessively.
In such cases, if the behavioral weight vector $\V{\hat{b}}_f^{k,l}$ is not confident, i.e., weights of behavioral categories are close to each other, one may want to decrease its contribution to the voting.
To achieve this, we propose two optional approaches; penalizing the behavioral weights with the entropy of the ``probability distribution'' $\V{\hat{b}}_f^{k,l}$, or with the uncertainty.
The contribution of votes of $\exptAnn_l$ to the committee formed for $\exptUnann_k$ is $\V{v}_f^{k,l} = \qmatrix{v_{f,1}^{k,l}, \cdots, v_{f,K}^{k,l}}$, and is given by
\begin{equation}
	v_{f,i}^{k,l} = \rbr{\log_2(K) - \entropy{\V{\hat{b}}_f^{k,l}}} \hat{b}_{f,i}^{k,l} \qbor \rbr{ 1 - \max_{1 \leq j \leq K} \hat{b}_{f,j}^{k,l}} \hat{b}_{f,i}^{k,l} \qbor \hat{b}_{f,i}^{k,l}.
\end{equation}
If $\max_i \hat{b}_{f,i}$ is close to $\sfrac{1}{K}$, which means that the computed vector weights the behaviors uniformly, then the factors $\rbr{\log_2(K) - \entropy{\V{\hat{b}}_f^{k,l}}}$ or $\rbr{ 1 - \max_{1 \leq j \leq K} \hat{b}_{f,j}^{k,l}}$ may be used to decrease the ``importance'' of the vote.

\subsubsection{Soft Voting}
\begin{equation}
	\hat{y}^k_f = \argmax_{1 \leq i \leq K} \sum_{l=1}^{\Ann{R}} v_{f,i}^{k,l}
\end{equation}

\subsubsection{Hard Voting}
\begin{equation}
	\hat{y}^k_f = \argmax_{1 \leq i \leq K} \Set{\argmax_{1 \leq j \leq K} v_{f,j}^{k,l}}{j=i, \, 1 \leq l \leq \Ann{R}}
\end{equation}

\chapter{Analyzing Behavioral Repertoire of Asleep Fruit Fly}

\chapter{Employing Proposed Pipeline for Collected Data}\label{chapter:employing-proposed-pipeline}

\input{8-Results}
\input{9-BastySoftwarePackage}
\input{10-Conclusion}

\bibliography{refs}

\end{document}
