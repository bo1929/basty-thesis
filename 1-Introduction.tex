\chapter{Introduction}\label{chapter:introduction}

\NOTE{
	Sleep is an essential behavioral program conserved across the animal kingdom.
	In mammals, sleep consists of multiple stages marked by physiological changes, including reductions in muscle tone and distinct electrophysiological activity patterns in the brain.
	In invertebrates, sleep has largely been studied as a unitary process and identified by bouts of consolidated immobility.
	Thus, careful characterization of underlying changes in behavior and physiology is needed to characterize distinct sleep stages in powerful genetic model systems such as Drosophila.
}

\section{Related Works}\label{section:related-works}
\NOTE{
	\begin{itemize}
		\item Computational {Neuroethology}: {A} {Call} to {Action}: \citep{datta_computational_2019}
		\item Quantifying behavior to understand the brain: \citep{pereira_quantifying_2020}
		\item Toward a {Science} of {Computational} {Ethology}: \citep{anderson_toward_2014}
	\end{itemize}
}
\NOTE{
	\begin{itemize}
		\item A deep sleep stage in {Drosophila} with a functional role in waste clearance: \citep{van_alphen_deep_2021}
		\item Most sleep does not serve a vital function: {Evidence} from {Drosophila} melanogaster: \citep{geissmann_most_2019}
		\item Automated analysis of long-term grooming behavior in {Drosophila} using a k-nearest neighbors classifier: \citep{qiao_automated_2018}
		\item Mapping the stereotyped behaviour of freely moving fruit flies: \citep{berman_mapping_2014}
		\item Systematic exploration of unsupervised methods for mapping behavior: \citep{todd_systematic_2017}
	\end{itemize}
}
\NOTE{
	\begin{itemize}
		\item Predictability and hierarchy in \textit{{Drosophila}} behavior: \citep{berman_predictability_2016}
		\item The manifold structure of limb coordination in walking {Drosophila}: \citep{deangelis_manifold_2019}
		\item The 103,200-arm acceleration dataset in the {UK} {Biobank} revealed a landscape of human sleep phenotypes: \citep{katori_103200-arm_2022}
	\end{itemize}
}
\NOTE{
	\begin{itemize}
		\item A dictionary of behavioral motifs reveals clusters of genes affecting {Caenorhabditis} elegans locomotion: \citep{brown_dictionary_2013}
		\item Deconstructing {Hunting} {Behavior} {Reveals} a {Tightly} {Coupled} {Stimulus}-{Response} {Loop}: \citep{mearns_deconstructing_2020}
		\item Continuous {Whole}-{Body} {3D} {Kinematic} {Recordings} across the {Rodent} {Behavioral} {Repertoire} - {SI}: \citep{marshall_continuous_2021}
		\item Mapping {Sub}-{Second} {Structure} in {Mouse} {Behavior}: \citep{wiltschko_mapping_2015}
		\item The {Mouse} {Action} {Recognition} {System} ({MARS}) software pipeline for automated analysis of social behaviors in mice: \citep{segalin_mouse_2021}
		\item Neural control of affiliative touch in prosocial interaction: \citep{wu_neural_2021}
		\item Partitioning variability in animal behavioral videos using semi-supervised variational autoencoders: \citep{whiteway_partitioning_2021}
		\item Structure of the {Zebrafish} {Locomotor} {Repertoire} {Revealed} with {Unsupervised} {Behavioral} {Clustering}: \citep{marques_structure_2018}
	\end{itemize}
}

\NOTE{
	\begin{itemize}
		\item {DeepEthogram}, a machine learning pipeline for supervised behavior classification from raw pixels: \citep{bohnslav_deepethogram_2021}
		\item B-{SOiD}, an open-source unsupervised algorithm for identification and fast prediction of behaviors: \citep{hsu_b-soid_2021}
		\item Ethoscopes: {An} open platform for high-throughput ethomics: \citep{geissmann_ethoscopes_2017}
		\item {JAABA}: interactive machine learning for automatic annotation of animal behavior: \citep{kabra_jaaba_2013}
		\item Simple {Behavioral} {Analysis} ({SimBA}) – an open source toolkit for computer classification of complex social behaviors in experimental animals: \citep{nilsson_simple_2020}
	\end{itemize}
}
\NOTE{
	\begin{itemize}
		\item Mapping {Sub}-{Second} {Structure} in {Mouse} {Behavior}
		\item 	{DeepLabCut}: markerless pose estimation of user-defined body parts with deep learning: \citep{mathis_deeplabcut_2018}
		\item Fast animal pose estimation using deep neural networks: \citep{pereira_fast_2019}
		\item {SLEAP}: {A} deep learning system for multi-animal pose tracking: \citep{pereira_sleap_2022}
	\end{itemize}
}

\section{Challenges}\label{section:challanges}

\section{Contributions}
