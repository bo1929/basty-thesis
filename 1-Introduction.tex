\chapter{Introduction}\label{chapter:introduction}

Sleep is an essential behavioral program conserved across the animal kingdom.
In mammals, sleep consists of multiple stages marked by physiological changes, including reductions in muscle tone and distinct electrophysiological activity patterns in the brain.
In invertebrates, sleep has largely been studied as a unitary process and identified by bouts of consolidated immobility.
Thus, careful characterization of underlying changes in behavior and physiology is needed to characterize distinct sleep stages in powerful genetic model systems such as Drosophila.

\section{Challenges}\label{section:challanges}

\section{Contributions}
