\clearpage\pagebreak
\begin{center}
	\MakeUppercase{\textbf{Özet}} \\ [3\baselineskip]
	\MakeUppercase{\thesistitletr} \\ [3\baselineskip]
	\MakeUppercase{\student} \\[\baselineskip]
	\MakeUppercase{\majortr \degreetr Tez\.{ı}, \monthtr~\year} \\[\baselineskip]
	Tez Danışmanı: \advisortr \\
	İkincil Tez Danışmanı: \coadvisortr \\
	[2\baselineskip]
	Anahtar Kelimeler: Uyku, Hesaplamalı etoloji, Davranış analizi, Aktivite tespiti, Davranış tasnifi
\end{center}

\singlespacing

Uyku, hayvanlar aleminde evrimsel olarak korunmuş, elzem bir davranış biçimidir.
Uykunun işlevlerini anlamak için, Drosophila Melanogaster gibi model organizmalarda uyku esnasında gözlemlenen davranışsal ve fizyolojik değişikliklerin başarılı bir şekilde karakterize edilmesi gerekmektedir.
Makine öğrenimini alanındaki gelişmeler, vücut bölümlerinin otomatik olarak takip edilebilmesini ve yüksek başarımlı poz tahmini yapılmasını sağlamıştır.
Ancak, davranışların tespit edilmesi ve sınıflandırılması, pozların ve pozlarda meydana gelen değişiklerin hangi davranış kategorilerine karşılık geldiğini hesaplamayı gerektirir.
Uyku esnasında sergilenen davranışların tespiti ve başarılı bir şekilde tasnif edilmesi, kendine özgü zorluklara sahiptir.
Mevcut yöntemler ve veri işleme yaklaşımları, uyku sırasındaki meydana gelen ve fark etmesi zor hareketlerden ziyade, makro ölçekte gerçekleşen postural değişikliklere odaklanarak geliştirilmiştir.
Uykuda gerçekleşen davranışları analiz etme hedefimiz, uzun uyku döngüleri sırasında seyrek olarak meydana gelen küçük değişiklikleri başarılı bir şekilde saptamayı ve sınıflandırmayı gerektirir.
Bu amaçla açık kaynak kodlu ve kullanımı kolay bir yazılım paketi olarak sunduğumuz bir veri işleme modeli olan basty’i geliştirdik.
Modelimiz, anlamlı davranış temsillerini hesaplama, uzun uyku deneylerinde (14-16 saat) aktiviteleri tespit etme ve düşük boyutlu davranışsal gömme uzaylarında en yakın komşu çözümlemesi gibi çeşitli analizlere olanak tanır.
Modelimizi, beş davranış kategorisine odaklanarak, uyku yoksunluğu ve vahşi tip uyku deneylerinin verileri ile değerlendirdik.
Sonuçlar, geliştirdiğimiz yaklaşımın davranış kategorilerini başarılı bir şekilde tasnif ettiğini, ve 0,8 AUC puanı elde edebildiğini gösteriyor.
Üstelik, yöntemimizin daha önce gözlemlenmeyen davranışları ve davranışsal repertuvarlardaki farklılıkları da tespit edebildiğini de ortaya koyduk.
Ayrıca, hem vahşi tip uyku hem de uyku yoksunluğu deneylerinde, davranışların uzamsal-zamansal niteliklerini ve zamana bağlı organizasyonlarını inceleyerek uyku sırasında sergilenen davranışsal repertuvarın bir analizini sunuyoruz.

\onehalfspacing
