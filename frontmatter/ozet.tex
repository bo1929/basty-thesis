\clearpage\pagebreak
\begin{center}
	\MakeUppercase{\textbf{Özet}} \\ [3\baselineskip]
	\MakeUppercase{\thesistitletr} \\ [3\baselineskip]
	\MakeUppercase{\student} \\[\baselineskip]
	\MakeUppercase{\majortr \degreetr Tez\.{ı}, \monthtr~\year} \\[\baselineskip]
	Tez Danışmanı: \advisortr \\
	İkincil Tez Danışmanı: \coadvisortr \\
	[2\baselineskip]
	Anahtar Kelimeler: Uyku, Hesaplamalı etoloji, Davranış analizi, Davranış tasnifi
\end{center}

\singlespacing

Uyku, hayvanlar aleminde evrimsel olarak korunmuş, elzem bir davranış biçimidir.
Uykunun işlevlerini anlamak için, \textit{Drosophila Melanogaster} gibi numune organizmalarda, uyku esnasında gözlemlenen davranışların ve fizyolojik değişikliklerin başarılı bir şekilde nitelendirilmesi gerekmektedir.
Makine öğrenmesi alanındaki gelişmeler, vücut bölümlerinin otomatik olarak takip edilebilmesini, yüksek başarımlı konum ve duruş tahmini yapılabilmesini sağlamıştır.
Ancak, davranışların tespit ve tasnif edilmesi, duruşların ve duruşlarda meydana gelen değişiklerin hangi davranış sınıflarına karşılık geldiğini hesaplamayı gerektirir.
Uyku esnasında sergilenen davranışların tespit ve tahlil edilmesi, kendine özgü zorluklara sahiptir.
Mevcut yöntemler ve veri işleme yaklaşımları, uyku sırasında meydana gelen fark edilmesi zor hareketlerden ziyade, büyük ölçekte gerçekleşen hareketlere ve duruş değişikliklerine odaklanarak geliştirilmiştir.
Uykuda meydana gelen davranışları tahlil etme hedefimiz, uzun uyku döngüleri sırasında seyrek olarak meydana gelen asgarî değişiklikleri yüksek başarım ile saptamayı ve sınıflandırmayı gerektirir.
Bu hedefle açık kaynaklı ve kullanımı kolay bir yazılım olarak sunduğumuz bir veri işleme uygulaması olan \texttt{basty}’i geliştirdik.
Yaklaşımımız, anlamlı davranış temsillerini hesaplama, uzun uyku deneylerinde (14-16 saat) faâliyetleri tespit etme ve düşük boyutlu gömme uzaylarında en yakın komşu çözümlemesi gibi çeşitli aşamalardan oluşmaktadır.
\texttt{basty}'i, beş davranış sınıfına odaklanarak, uyku yoksunluğu ve vahşi tip uyku deneylerinin verileri ile değerlendirdik.
Sonuçlar, geliştirdiğimiz yaklaşımın davranış sınıflarını başarılı bir şekilde tasnif ettiğini ve 0,8 AUC puanı elde edebildiğini göstermektedir.
Üstelik, yöntemimizin daha önce gözlemlenmeyen davranışları ve davranış repertuvarlarındaki farklılıkları tespit edebildiğini de ortaya koyduk.
Ayrıca, hem vahşi tip uyku hem de uyku yoksunluğu deneylerinde, davranışların uzam-zaman eksenindeki niteliklerini ve uyku esnasındaki tertiplerini inceleyerek uyku sırasında sergilenen davranış repertuvarın bir incelemesini de sunuyoruz.

\onehalfspacing
