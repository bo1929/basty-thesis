\setlength{\parindent}{0pt}
\chapter{Introduction}\label{chapter:introduction}
Sleep is an essential behavioral program conserved across the animal kingdom, in diverse species ranging from jellyfish to humans, whose function remains unknown \citep{campbell_animal_1984, nath_jellyfish_2017}.
In mammals, sleep consists of multiple stages marked by physiological changes, including reductions in muscle tone and distinct electrophysiological activity patterns in the brain \citep{corner_sleep_1977, sauer_dynamics_2003} In invertebrates, sleep has largely been studied as a unitary process and identified by bouts of consolidated immobility.
Thus, careful characterization of underlying changes in behavior and physiology is needed for understanding the functional role of the sleep, and characterizing distinct sleep stages in powerful genetic model systems such as \textit{Drosophila Melanogaster}.

A flurry of recent technical advances enabled automated quantification of behaviors, and this has opened up a new field of ``computational ethology'' \citep{anderson_toward_2014, datta_computational_2019}.
Particularly, the recent progress in deep learning lead to emergence of  methods for tracking animal motion, which provided the opportunity of studying naturalistic behavior at a resolution not previously imaginable \citep{pereira_quantifying_2020}.
Developments of pose estimation and tracking tools such as DeepLabCut \citep{mathis_deeplabcut_2018} and SLEAP \citep{pereira_fast_2019, pereira_sleap_2022} collecting spatio-temporal data about the animals possible.
However, this data only consists coordinate values in two or three dimensional spatial domain, depending on the employed tracking tool.
Quantification of rich and complex behavioral repertoires of the animals, considering its temporal structure and ambiguity, is a fundamentally difficult problem without a clear ground truth, as discussed in \citet{pereira_quantifying_2020}.

\NOTE{
	\begin{itemize}
		\item A deep sleep stage in {Drosophila} with a functional role in waste clearance: \citep{van_alphen_deep_2021}
		\item Most sleep does not serve a vital function: {Evidence} from {Drosophila} melanogaster: \citep{geissmann_most_2019}
		\item Automated analysis of long-term grooming behavior in {Drosophila} using a k-nearest neighbors classifier: \citep{qiao_automated_2018}
		\item Mapping the stereotyped behaviour of freely moving fruit flies: \citep{berman_mapping_2014}
		\item Systematic exploration of unsupervised methods for mapping behavior: \citep{todd_systematic_2017}
	\end{itemize}
}
\NOTE{
	\begin{itemize}
		\item Predictability and hierarchy in \textit{{Drosophila}} behavior: \citep{berman_predictability_2016}
		\item The manifold structure of limb coordination in walking {Drosophila}: \citep{deangelis_manifold_2019}
		\item The 103,200-arm acceleration dataset in the {UK} {Biobank} revealed a landscape of human sleep phenotypes: \citep{katori_103200-arm_2022}
	\end{itemize}
}
\NOTE{
	\begin{itemize}
		\item A dictionary of behavioral motifs reveals clusters of genes affecting {Caenorhabditis} elegans locomotion: \citep{brown_dictionary_2013}
		\item Deconstructing {Hunting} {Behavior} {Reveals} a {Tightly} {Coupled} {Stimulus}-{Response} {Loop}: \citep{mearns_deconstructing_2020}
		\item Continuous {Whole}-{Body} {3D} {Kinematic} {Recordings} across the {Rodent} {Behavioral} {Repertoire} - {SI}: \citep{marshall_continuous_2021}
		\item Mapping {Sub}-{Second} {Structure} in {Mouse} {Behavior}: \citep{wiltschko_mapping_2015}
		\item The {Mouse} {Action} {Recognition} {System} ({MARS}) software pipeline for automated analysis of social behaviors in mice: \citep{segalin_mouse_2021}
		\item Neural control of affiliative touch in prosocial interaction: \citep{wu_neural_2021}
		\item Partitioning variability in animal behavioral videos using semi-supervised variational autoencoders: \citep{whiteway_partitioning_2021}
		\item Structure of the {Zebrafish} {Locomotor} {Repertoire} {Revealed} with {Unsupervised} {Behavioral} {Clustering}: \citep{marques_structure_2018}
	\end{itemize}
}

\NOTE{
	\begin{itemize}
		\item {DeepEthogram}, a machine learning pipeline for supervised behavior classification from raw pixels: \citep{bohnslav_deepethogram_2021}
		\item B-{SOiD}, an open-source unsupervised algorithm for identification and fast prediction of behaviors: \citep{hsu_b-soid_2021}
		\item Ethoscopes: {An} open platform for high-throughput ethomics: \citep{geissmann_ethoscopes_2017}
		\item {JAABA}: interactive machine learning for automatic annotation of animal behavior: \citep{kabra_jaaba_2013}
		\item Simple {Behavioral} {Analysis} ({SimBA}) – an open source toolkit for computer classification of complex social behaviors in experimental animals: \citep{nilsson_simple_2020}
	\end{itemize}
}
