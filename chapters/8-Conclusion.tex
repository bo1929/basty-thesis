\setlength{\parindent}{0pt}
\chapter{\bf CONCLUSION}\label{chapter:conclusion}

% Retell what you have done.
Ethology, the scientific study of animal behavior, focuses on behavior under natural conditions, and is as a rapidly growing field.
With its scientific roots in late 19th and 20th centuries, it became a well-recognized scientific discipline and yielded significant insight into general principles of the nervous system.
Behavior, as arguably the most robust output of the brain, is needed to be deciphered to understand complex neurobiological phenomena, including sleep.
Technical advances, ranging from imaging technologies to machine learning techniques, transformed ethology into a new computational discipline, and the field of computational ethology has emerged.

In the past decade, remarkable tools, which involve computer vision and machine learning, have been developed for automated quantification of animal behavior.
Employing such tools provided unprecedented insight into complex behaviors, such as hunting and mating \citep{mearns_deconstructing_2020, janisch_deciphering_2021}.
Yet still, challenges remain open.
While behaviors exhibited by awake animals tend to be characterized by major postural and positional changes, behavioral repertoire of asleep animals contains subtle movements and unobtrusive changes of a small set of body parts, that sparsely occur during long sleep cycles.
Thus,

To this end,

We evaluated,

% Limitations.

% Future work.


% How this research enable other research, sleep phenotyping, molecular biology experiments to identify genetic programs etc.
