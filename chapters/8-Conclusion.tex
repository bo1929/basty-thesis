\setlength{\parindent}{0pt}
\chapter{\bf CONCLUSION}\label{chapter:conclusion}

% Retell what you have done.
Ethology, the scientific study of animal behavior, focuses on behavior under natural conditions and is a rapidly growing field.
With its scientific roots in the late 19th and 20th centuries, it became a well-recognized scientific discipline and yielded significant insight into general principles of the nervous system.
Behavior, as arguably the most robust output of the brain, is needed to be deciphered to understand complex neurobiological phenomena, including sleep.
Technical advances, ranging from imaging technologies to machine learning techniques, transformed ethology into a new computational discipline, and the field of computational ethology has emerged.

In the past decade, remarkable tools, which involve computer vision and machine learning, have been developed for automated quantification of animal behavior.
Employing such tools provided unprecedented insight into complex behaviors, such as hunting and mating \citep{mearns_deconstructing_2020, janisch_deciphering_2021}.
Yet still, challenges remain open.
While behaviors exhibited by awake animals tend to be characterized by major postural and positional changes, the behavioral repertoire of asleep animals includes subtle movements and unobtrusive changes that sparsely occur during long sleep cycles.
Thus, exploring and quantifying the behavioral repertoire of sleep requires special care and additional considerations.

To this end, we develop a multi-stage and end-to-end behavior mapping pipeline, specifically designed for animals in dormancy state, and release it as a configurable and easy-to-use software package, called \texttt{basty}.
Our approach consists of computing representations capturing complex spatio-temporal characteristics of behaviors, detecting micro-activities exhibited in long-duration sleep bouts, and mapping detected activities utilizing a nearest neighbor based analysis on the low-dimensional behavioral embeddings.
Our pipeline is able to handle noisy pose estimation data, with missing points (i.e., occlusions), through rigorous pre-processing and post-processing steps.
Using novel approaches such as semi-supervised pair embeddings approach and experiment committee by voting, \texttt{basty} can tackle mapping of the rich behavioral repertoire of dormant fruit fly with limited supervision.

We evaluate our \texttt{basty} with a dataset of sleep deprivation and wild-type sleep experiments, focusing on five behavioral categories: grooming, feeding, proboscis pumping, haltere switch, and postural adjustments.
Automated quantification of proboscis pumping and haltere switch is achieved for the first time in this work.
Such behaviors are potentially related to behaviorally defined states of sleep similar to the REM sleep (rapid eye movement sleep) phase observed in mammals and birds.
Results demonstrate that our pipeline robustly detects activities, and successfully maps behavioral categories.
Overall, \texttt{basty} achieves a macro average of 0.8 AUC score.
Moreover, we demonstrate its ability to detect unobserved behaviors and differences between behavioral repertoires.

% Limitations.
Although we focus on five behavioral categories, in fact, behavioral repertoire of dormant Drosophila Melanogaster is much richer. In fact,

3D tracking


% Future work.

% How does this research enables other research, sleep phenotyping, molecular biology experiments to identify genetic programs, etc.
