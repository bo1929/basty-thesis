\setlength{\parindent}{0pt}
\chapter{\bf CONCLUSION}\label{chapter:conclusion}

% Retell what you have done.
Ethology, the scientific study of animal behavior, focuses on behavior under natural conditions and is a rapidly growing field.
With its scientific roots in the late 19th and 20th centuries, it became a well-recognized scientific discipline and yielded significant insight into general principles of the nervous system.
Behavior, as arguably the most robust output of the brain, is needed to be deciphered to understand complex neurobiological phenomena, including sleep.
Technical advances, ranging from imaging technologies to machine learning techniques, transformed ethology into a new computational discipline, and the field of computational ethology has emerged.

While behaviors exhibited by awake animals tend to be characterized by major postural and positional changes, the behavioral repertoire of asleep animals contains subtle movements and unobtrusive changes that sparsely occur during long sleep cycles.

% Limitations.

% Future work.


% How does this research enables other research, sleep phenotyping, molecular biology experiments to identify genetic programs, etc.
