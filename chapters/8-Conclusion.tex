\setlength{\parindent}{0pt}
\chapter{\bf CONCLUSION}\label{chapter:conclusion}

Ethology, the scientific study of animal behavior, focuses on behavior under natural conditions, and it is a rapidly growing field. With its scientific roots in the late 19th and early 20th centuries, it gradually became a well-recognized research area and yielded significant insight into general principles of the nervous system.
Behaviors are arguably the most robust outputs of the brain. The technical advances, ranging from imaging technologies to machine learning techniques, has recently transformed ethology into a new computational discipline.

In the past decade, remarkable tools based on computer vision and machine learning have been developed for automated quantification of animal behavior.
The use of such tools in analyzing animal behaviors has provided unprecedented insight into complex behaviors, such as hunting and mating \citep{mearns_deconstructing_2020, janisch_deciphering_2021}.
Yet, many questions remain unanswered.
While behaviors exhibited by awake animals tend to be characterized by major postural and positional changes, the sleeping animals go through subtle movements and unobtrusive postural changes that sporadically  occur during long sleep cycles.
Thus, exploring and quantifying the behavioral repertoire of sleep requires special care and additional considerations.

To this end, we developed a multi-stage and end-to-end behavior mapping pipeline specifically designed for animals in the dormancy state. We release it as a freely accessible, configurable, and easy-to-use software package called \texttt{basty}.
Our approach is to compute the representations for capturing complex spatio-temporal characteristics of behaviors, detecting micro-activities exhibited in long-duration sleep bouts, and mapping detected activities utilizing a nearest neighbor analysis on the behavioral embeddings.
Our pipeline can handle noisy pose estimation data, with missing points (i.e., occlusions), through rigorous pre-processing and post-processing.
Using novel approaches such as semi-supervised pair embeddings and experiment committee by voting, \texttt{basty} can tackle mapping of the rich behavioral repertoire of dormant fruit flies with limited supervision.

We evaluate our \texttt{basty} with a dataset of sleep deprivation and wild-type sleep experiments, focusing on five behavioral categories: \Grooming, \Feeding, \ProboscisPumping, \HaltereSwitch, and \PosturalAdjustment.
Automated quantification of \ProboscisPumping and \HaltereSwitch is achieved for the first time in this work.
Such behaviors are potentially related to behaviorally defined states of sleep, similar to the REM sleep (rapid eye movement sleep) phase observed in mammals and birds.
For example, \citet{van_alphen_deep_2021} claims that \ProboscisPumping is accompanied by decreased brain activity and raised arousal thresholds, which might indicate a deep sleep state.
Results demonstrate that our pipeline robustly detects activities and successfully maps behavioral categories.
Overall, \texttt{basty} achieves a macro average of 0.8 AUC score.
Moreover, we demonstrate its ability to detect unobserved behaviors and differences between behavioral repertoires.

Although we focus on 5 behavioral categories, in fact, the behavioral repertoire of dormant \textit{Drosophila Melanogaster} is much richer. There exist behaviors that do not fall into any of 5 the defined categories.
As behaviors need to be annotated, the problem of ``lumping'' versus ``splitting'' emerges due to the nature of behaviors. For instance, we lump face \Grooming and abdomen \Grooming under a single category of \Grooming.
Therefore, a future direction is extending the considered behavioral categories and even splitting existing categories into finer subcategories.

The approach of segmentation of continuous and hierarchical behaviors into categorical components is necessarily human-biased and does not reflect the true nature of behaviors.
A sequence of simple behaviors together might hierarchically generate compositional behaviors.
Similarly, two behaviors might accompany each other, resulting in a composite behavior.
One can utilize behavioral scores to analyze such structures; however, in its current state, our pipeline does not directly deal with composite and compositional behaviors.
Hence, \texttt{basty} can be extended to analyze such behaviors in the future.

Popular pose estimation tools, such as DeepLabCut and SLEAP, perform tracking on the videos recorded with a single camera, and hence, the inference is made in the two-dimensional projection.
The resulting pose estimation data do not directly reflect the true spatial characteristics and contain many missing points due to occlusions.
However, the true spatial characteristics of behaviors can only be captured in three-dimensional space.
Our pipeline, \texttt{basty}, can easily be configured to operate on three-dimensional pose values as three-dimensional pose estimation tools become  increasingly available.
