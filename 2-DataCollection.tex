\chapter{Experiments and Data Collection}\label{chapter:expt-data-collection}
In this section, we describe how the sleep experiments are conducted and fly video recordings data were collected.

\section{Sleep Experiments and Video Recordings}

\textit{The experimental data were collected by Dr. Mehmet Fatih Keles at Johns Hopkins University.}

A custom imaging setup was used to perform high-resolution characterization of sleep-related behaviors in flies. This set up includes a custom 3D printed chamber (7.2X4.3X2.4 mm [WxHxL]) that is placed in front of an IR sensitive (Flir) 30 FPS camera with telecentric lens (Edmund Optics).

\begin{figure}[ht!]
	\centering
	\includegraphics[width=0.75\linewidth]{figures/ExperimentalSetup.pdf}
	\caption[An illustration of the experimental setup which is used to perform high-resolution imaging of experiments.]{An illustration of the experimental setup which is used to perform high-resolution imaging of experiments.}
\end{figure}

Flies recorded from ZT10-ZT2 (16 hours) total.
Each chamber has a food port (1.5 mm diameter) that allows access to liquid food (2.5\% yeast, 2.5\% sugar).
Recording setup is in a light tight box and humidity control (60\%) is achieved via a humidifier plugged into a humidity control switch. Experimental flies are loaded to individual chambers at ZT8-ZT9 via mouth pipette or a small vacuum pump.
Individual chambers are sealed with a 7×7 mm acrylic windows.
Windows are coated with SigmaCote to prevent flies from ventral or dorsal postural positions. 5-7 day old female and male flies are used in the experiments.

\section{Pose Estimation and Tracking}
A recently developed deep neural network based software, DeepLabCut was used to achieve markerless pose estimation.
Over 20 body parts are first labeled in 1654 images from 28 animals (16 female, 12 male) to train the model with a 95/5 train/test split.
We used a ResNet-50 based neural network with a batch size of 4 and 200k training iterations.
Rest of the settings were kept default.
The resulting network has a test error of 3.67 pixels.

\begin{figure}[ht!]
	\centering
	\includegraphics[width=0.75\linewidth]{figures/FlyTrackedBodyParts.png}
	\caption[An example frame of the fruit fly placed in 3D printed chamber.] {An example video recording frame of the fruit fly placed in 3D printed chamber. Colorful markers indicate the tracked body parts.}
\end{figure}

\section{Behavior Annotations}

3 human annotators labeled 5 different behaviors (feeding, grooming, moving, haltere switch, proboscis pumping) across 11 videos (14 hours and 16 hours each, respectively for 7 wild type sleep experiments and 4 sleep deprivation experiments).
A single experiment is annotated by all 3 annotators to check rigor and overlap among annotators. We only used the annotations of a fly on when  annotators agreed at least on the 90\% of the labels.

\begin{figure}[ht!]
	\centering
	\begin{subfigure}[ht!]{0.95\linewidth}
		\centering\includegraphics[width=\linewidth]{figures/FlyF11-01182022175505_annotation_ethogram.pdf}
		\caption{Wild type.}
	\end{subfigure}%

	\centering
	\begin{subfigure}[ht!]{0.95\linewidth}
		\centering\includegraphics[width=\linewidth]{figures/FlyF52SD-11242021161943_annotation_ethogram.pdf}
		\caption{Sleep-deprived.}
	\end{subfigure}%
	\caption[Two example ethograms of annotated behaviors observed during the sleep experiments.]{Two example ethograms of annotated behaviors observed during the sleep experiments. Dark period starts at ZT12, and ends at ZT0.}
\end{figure}
